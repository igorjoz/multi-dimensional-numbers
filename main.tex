\documentclass[aspectratio=169,11pt]{beamer}

% --- język, polskie ogonki ---
\usepackage[polish]{babel}
\usepackage[utf8]{inputenc}
\usepackage[T1]{fontenc}

% --- trochę ładniej ---
\usetheme{Madrid}
\usecolortheme{default}
\usefonttheme{professionalfonts}
\setbeamertemplate{navigation symbols}{}

% matematyka
\usepackage{amsmath, amssymb, amsthm}
\usepackage{bm}

% grafiki
\usepackage{hyperref}
\usepackage{graphicx}
\usepackage{svg}
\graphicspath{{./}{./img/}{./images/}}

% tabelki
\usepackage{booktabs}
\usepackage{array}
\newcolumntype{C}{>{\centering\arraybackslash}m{1.7cm}}

% żeby małe napisy pod wzorami wyglądały normalnie
\newcommand{\tinyit}[1]{\tiny\textit{#1}}

% -------------------------------------------------
% Informacje tytułowe
% -------------------------------------------------
\title[Liczby wielowymiarowe]{Liczby wielowymiarowe}
\author{Igor Józefowicz}
\institute{Politechnika Gdańska}
\date{3 listopada 2025}

% =================================================
\begin{document}
% =================================================

\frame{\titlepage}

% -------------------------------------------------
\begin{frame}{Plan prezentacji}
  \tableofcontents
\end{frame}

% (przeniesiono slajd Broom Bridge za "Krótka historia")

% =================================================
\section{Motywacja i historia}
% =================================================

\begin{frame}{Dlaczego w ogóle ``liczby wielowymiarowe''?}
  \begin{itemize}
    \item Liczby rzeczywiste $\mathbb{R}$ opisują długości, pomiary, ciągłe wartości.
    \item Ale równanie $x^2 + 1 = 0$ \textbf{nie ma} rozwiązania w $\mathbb{R}$.
    \item Dodajemy więc nowy obiekt $i$ taki, że $i^2=-1$.
    \item Dostajemy liczby zespolone $\mathbb{C}$, które są już \textbf{dwuwymiarowe}.
    \item Potem spróbujemy pójść dalej: 4 wymiary (kwaterniony), 8 wymiarów (oktoniony).
  \end{itemize}

  \vspace{0.1cm}
  \begin{center}
    \includegraphics[width=0.4\textwidth]{images/imaginary-numbers.jpeg}
    \\
    {\tinyit{Źródło: \url{https://www.thepromptmag.com/wp-content/uploads/2018/09/imaginary-numbers.jpeg}}}
  \end{center}
\end{frame}

\begin{frame}{Krótka historia}
  \begin{itemize}
    \item XVIII w.: Euler, Gauss -- zdefiniowanie liczb zespolonych.
    \item 1843: William Rowan Hamilton wprowadza \textbf{kwaterniony} $\mathbb{H}$.
    \item 1845: Cayley i Graves konstruują \textbf{oktoniony} $\mathbb{O}$.
  \end{itemize}

  \vspace{0.2cm}
  \begin{columns}[T,totalwidth=\textwidth]
    \begin{column}{0.33\textwidth}
      \centering
      \includegraphics[height=0.35\textheight,keepaspectratio]{Leonhard_Euler_-_Jakob_Emanuel_Handmann_(Kunstmuseum_Basel).jpg}\\[2pt]
      {\scriptsize Euler}

      \vspace{2pt}
      {\tinyit{Źródło: \url{https://en.wikipedia.org/wiki/File:Leonhard_Euler_-_Jakob_Emanuel_Handmann_(Kunstmuseum_Basel).jpg}}}
    \end{column}
    \begin{column}{0.33\textwidth}
      \centering
      \includegraphics[height=0.35\textheight,keepaspectratio]{Carl_Friedrich_Gauss.jpg}\\[2pt]
      {\scriptsize Gauss}

      \vspace{2pt}
      {\tinyit{Źródło: \url{https://commons.wikimedia.org/wiki/File:Carl_Friedrich_Gauss.jpg}}}
    \end{column}
    \begin{column}{0.33\textwidth}
      \centering
      \includegraphics[height=0.35\textheight,keepaspectratio]{William_Rowan_Hamilton_portrait_oval_combined.png}\\[2pt]
      {\scriptsize Hamilton}

      \vspace{2pt}
      {\tinyit{Źródło: \url{https://commons.wikimedia.org/wiki/File:William_Rowan_Hamilton_portrait_oval_combined.png}}}
    \end{column}
  \end{columns}
\end{frame}

\begin{frame}{Hamilton i Broom Bridge (1843)}
  	\textbf{Anegdota:} 16 października 1843 Hamilton wyrył wzór
  $i^2 = j^2 = k^2 = ijk = -1$
  na moście Broom Bridge w Dublinie.

  \vspace{0.3cm}
  \begin{center}
    \includegraphics[width=0.5\textwidth]{broom_bridge_plaque.jpg}\\
    {\tinyit{Źródło: \url{https://upload.wikimedia.org/wikipedia/commons/5/53/Broom_bridge_plaque.jpg}}}
  \end{center}
\end{frame}


% =================================================
\section{Liczby zespolone $\mathbb{C}$}
% =================================================

\begin{frame}{Definicja liczb zespolonych}
  \begin{columns}[T]
    \begin{column}{0.5\textwidth}
      \begin{block}{Definicja}
        $\displaystyle
        \mathbb{C} = \{ a + bi : a,b \in \mathbb{R}, \ i^2 = -1 \}.
        $
      \end{block}

      \begin{itemize}
        \item $a$ -- część rzeczywista, $\Re(z)$,
        \item $b$ -- część urojona, $\Im(z)$.
        \item Każde $z = a+bi$ to punkt $(a,b)$ na płaszczyźnie.
      \end{itemize}

      \vspace{0.1cm}
      \textbf{Wniosek:} $\mathbb{C}$ jest \textbf{2-wymiarową} przestrzenią nad $\mathbb{R}$.

      \vspace{0.1cm}
      \textbf{Sprzężenie:} $\overline{a+bi} = a-bi$.\\
      \textbf{Moduł:} $|a+bi| = \sqrt{a^2+b^2}$.
    \end{column}
    \begin{column}{0.5\textwidth}
      \centering
      \includesvg[width=0.8\textwidth]{Complex_number_illustration.svg}\\[5pt]
      \tinyit{Źródło: \url{https://commons.wikimedia.org/wiki/File:Complex_number_illustration.svg}}
    \end{column}
  \end{columns}
\end{frame}

\begin{frame}{Działania w $\mathbb{C}$}
  Niech $z_1=a+bi$, $z_2=c+di$.
  \begin{itemize}
    \item Dodawanie:
    \[
    z_1+z_2 = (a+c) + (b+d)i
    \]
    \item Mnożenie:
    \[
    z_1 z_2 = (ac - bd) + (ad+bc)i
    \]
    \item Dzielenie:
    \[
    \frac{z_1}{z_2} =
    \frac{(a+bi)(c-di)}{c^2+d^2}
    = \frac{(ac+bd)+(bc-ad)i}{c^2+d^2}
    \]
  \end{itemize}

  \vspace{0.3cm}
  \textbf{Własność kluczowa:}
  \[
  |z_1 z_2| = |z_1|\,|z_2|
  \quad\text{oraz}\quad
  \arg(z_1 z_2) = \arg(z_1)+\arg(z_2).
  \]
  \tinyit{czyli mnożenie liczby zespolonej to: skalowanie + obrót na płaszczyźnie}
\end{frame}

\begin{frame}{Zastosowania liczb zespolonych}
  \begin{itemize}
    \item \textbf{Fizyka / sygnały}: prąd przemienny $I(t)=I_0 e^{i\omega t}$, impedancja, faza.
    \item \textbf{Mechanika kwantowa}: funkcja falowa $\psi(x)$ jest zespolona.
    \item \textbf{Grafika 2D}: obrót punktu $(x,y)$ o kąt $\theta$ to mnożenie przez $e^{i\theta}$.
  \end{itemize}

  \vspace{0.4cm}
  \textbf{Przykład obrotu o 90$^\circ$:}\\[4pt]
  $z = 1+0i$, \quad $e^{i\pi/2} = i$ \\[4pt]
  $z' = z \cdot i = i$\\
  Punkt $(1,0)$ przechodzi w $(0,1)$.
\end{frame}


% =================================================
\section{Kwaterniony $\mathbb{H}$}
% =================================================

\begin{frame}{Przejście do kwaternionów}
  Pytanie Hamiltona: \\
  \emph{czy da się zrobić coś jak liczby zespolone, ale dla pełnych obrotów 3D?}

  \vspace{0.3cm}
  Odpowiedź:
  \begin{block}{Definicja}
    $\displaystyle
    \mathbb{H} = \{ a + bi + cj + dk : a,b,c,d \in \mathbb{R} \}
  $
  gdzie jednostki $i,j,k$ spełniają
  \[
    i^2 = j^2 = k^2 = ijk = -1.
  \]
  \end{block}

  \begin{itemize}
    \item To już \textbf{4-wymiarowa} przestrzeń nad $\mathbb{R}$.
    \item Każdy kwaternion ma część skalarną ($a$) i wektorową ($(b,c,d)$).
  \end{itemize}
\end{frame}

\begin{frame}{Tabela mnożenia}
  Podstawowe reguły:
  \[
    ij = k,\quad jk = i,\quad ki = j,
  \]
  \[
    ji = -k,\quad kj = -i,\quad ik = -j.
  \]

  \vspace{0.3cm}
  Zauważ:
  \[
    ij \neq ji.
  \]

  \vspace{0.3cm}
  \textbf{Wniosek:} Mnożenie kwaternionów \textbf{nie jest przemienne}.

  \vspace{0.4cm}
  \textbf{Mini-przykład:}
  \[
    (1+i)(1+j)=1+i+j+k.
  \]

  \vspace{0.2cm}
  
\end{frame}

\begin{frame}{Działania w $\mathbb{H}$}
  Niech
  $
    q_1 = a+bi+cj+dk, \quad
    q_2 = e+fi+gj+hk.
  $
  \begin{itemize}
    \item Dodawanie: współrzędnie, jak w $\mathbb{R}^4$.
    \item Mnożenie: korzystamy z tabelki $ij=k$, $ji=-k$, itd.
    \item Sprzężenie:
    \[
      \overline{q} = a-bi-cj-dk.
    \]
    \item Norma:
    \[
      \lVert q\rVert = \sqrt{a^2+b^2+c^2+d^2}.
    \]
    \item Odwrotność (jeśli $q\neq 0$):
    \[
      q^{-1} = \frac{\overline{q}}{\lVert q\rVert^2}.
    \]
  \end{itemize}

  \vspace{0.2cm}
  \textbf{Ważne:} Każdy niezerowy kwaternion ma odwrotność.\\
  To znaczy, że w $\mathbb{H}$ można \textbf{dzielić}.
\end{frame}

\begin{frame}{Kwaterniony a rotacje 3D}
  \textbf{Cel praktyczny:} opisać obrót wektora 3D bez niestabilności numerycznych.

  \vspace{0.2cm}
  Kwaternion jednostkowy $q$ (tzn. $\lVert q\rVert=1$) reprezentuje pewien obrót
  w przestrzeni 3D.

  \vspace{0.2cm}
  Jeśli mamy wektor $\vec{v}$, traktujemy go jako czysto wektorowy kwaternion
  $0 + v_x i + v_y j + v_z k$ i liczymy:
  \[
    \vec{v}' = q \, \vec{v} \, q^{-1}.
  \]

  \vspace{0.4cm}
  \textbf{Własność kluczowa:} długość $\|\vec{v}\|$ się nie zmienia.
  To znaczy: to jest \emph{czysty obrót}, bez rozciągania.

  \vspace{0.4cm}
  Dlatego kwaterniony są używane w \textbf{Unity, Blenderze, silnikach gier,
  dronach, nawigacji satelitarnej}.
\end{frame}

\begin{frame}{Stabilność numeryczna / Euler vs kwaterniony}
  \begin{itemize}
    \item W opisie obrotu przez kąty Eulera pojawia się problem \textbf{gimbal lock}
    (dwie osie pokrywają się i tracimy jeden stopień swobody).
    \item Kąty Eulera są też podatne na kumulację błędu przy wielokrotnych
    obrotach/aktualizacjach.
    \item Kwaterniony jednostkowe można po prostu normalizować
    (sprowadzać z powrotem do długości 1)
    i zachować stabilność obrotu numerycznie.
  \end{itemize}

  \vspace{0.4cm}
  \textbf{W praktyce:} silniki gier i narzędzia 3D trzymają orientację
  obiektu właśnie jako kwaternion, nie jako $(yaw,pitch,roll)$.
\end{frame}

\begin{frame}{Kwaterniony jako macierze $2\times 2$ zespolone}
  Istnieje odwzorowanie
  \[
    a+bi+cj+dk
    \quad \longmapsto \quad
    \begin{pmatrix}
      a+bi & c+di \\
      -c+di & a-bi
    \end{pmatrix}
  \]

  \begin{itemize}
    \item Mnożenie kwaternionów odpowiada mnożeniu tych macierzy.
    \item Norma kwaternionu wiąże się z wyznacznikiem takiej macierzy.
    \item To połączenie pokazuje, że kwaterniony są głęboko związane
    z macierzami obrotu i ogólnie z algebrą macierzową.
  \end{itemize}

  \vspace{0.3cm}
  
\end{frame}


% =================================================
\section{Konstrukcja Cayley--Dicksona}
% =================================================

\begin{frame}{Podwajanie wymiaru: konstrukcja Cayley--Dicksona}
  \begin{itemize}
    \item Istnieje ogólny sposób budowania kolejnych systemów liczbowych
    przez \textbf{podwajanie wymiaru}.
    \item Zaczynamy od $\mathbb{R}$ (1 wymiar).
    \item Robimy z tego uporządkowaną parę liczb rzeczywistych:
    dostajemy $\mathbb{C}$ (2 wymiary).
    \item Potem pary liczb zespolonych $\Rightarrow$ kwaterniony $\mathbb{H}$ (4 wymiary).
    \item Potem pary kwaternionów $\Rightarrow$ oktoniony $\mathbb{O}$ (8 wymiarów).
  \end{itemize}

  \vspace{0.3cm}
  Ten proces nazywa się \textbf{konstrukcją Cayley--Dicksona}.
\end{frame}

\begin{frame}{Co tracimy przy każdym podwojeniu?}
  \begin{itemize}
    \item $\mathbb{R}$: uporządkowane, przemienne, łączne, można dzielić.
    \item $\mathbb{C}$: nie jest już ``uporządkowane'' w sensie zgodnym z mnożeniem,
    ale nadal przemienne i łączne, można dzielić.
    \item $\mathbb{H}$: \textbf{nie jest przemienne}, ale nadal łączne, można dzielić.
    \item $\mathbb{O}$: \textbf{nie jest przemienne ani łączne}, ale jest tzw.
    ``alternatywne'' i nadal można dzielić.
  \end{itemize}

  \vspace{0.4cm}
  Dalej (16 wymiarów: \textbf{sedeniony}) tracimy nawet możliwość dzielenia:
  pojawiają się dzielniki zera $\Rightarrow$ algebra psuje się dla zastosowań
  numerycznych.
\end{frame}

\begin{frame}{Tabela własności (schematycznie)}
  \scriptsize
  \centering
  \begin{tabular}{lC C C C C C}
    \toprule
     & \textbf{Wymiar} & \textbf{Uporządk.} & \textbf{Przem.} & \textbf{Łączne} & \textbf{Dzielenie} & \textbf{Brak dzieln. zera}\\
    \midrule
    $\mathbb{R}$ (rzeczywiste)      & 1  & tak & tak & tak & tak & tak \\
    $\mathbb{C}$ (zespolone)        & 2  & nie & tak & tak & tak & tak \\
    $\mathbb{H}$ (kwaterniony)      & 4  & nie & nie & tak & tak & tak \\
    $\mathbb{O}$ (oktoniony)        & 8  & nie & nie & nie & tak & tak \\
    sedeniony itd.                  & 16 & nie & nie & nie & \textbf{nie} & \textbf{nie} \\
    \bottomrule
  \end{tabular}

  \vspace{0.4cm}
  \tinyit{inspiracja: klasyczna tabela własności algebr Cayley--Dicksona}
\end{frame}


% =================================================
\section{Oktoniony (ciekawostka)}
% =================================================

\begin{frame}{Oktoniony w dwóch zdaniach}
  \begin{itemize}
    \item Oktoniony $\mathbb{O}$ są 8-wymiarowe i mają 7 ``jednostek urojonych''.
    \item Mnożenie nie jest już nawet łączne,
    ale ciągle można dzielić (nie ma dzielników zera).
  \end{itemize}

  \vspace{0.3cm}
  Zastosowania:
  \begin{itemize}
    \item teoretyczna fizyka wysokich energii, symetrie wyjątkowe (tzw. grupy wyjątkowe),
    \item matematyka czysto abstrakcyjna.
  \end{itemize}

  \vspace{0.3cm}
  W praktyce inżynierskiej (grafika 3D, robotyka, Unity, Blender): \textbf{używamy kwaternionów, nie oktonionów.}
\end{frame}


% =================================================
\section{Porównanie i podsumowanie}
% =================================================

\begin{frame}{Różnice w działaniach}
  \begin{itemize}
    \item W $\mathbb{C}$: mnożenie jest przemienne ($zw=wz$).
    \item W $\mathbb{H}$: mnożenie \textbf{nie} jest przemienne ($ij = k$, ale $ji=-k$),
    ale jest łączne.
    \item W $\mathbb{O}$: nawet łączność pada, tzn.
    $(ab)c \neq a(bc)$ w ogólności.
  \end{itemize}

  \vspace{0.4cm}
  \textbf{Intuicja:}
  Im większy wymiar (2,4,8,16,\dots), tym więcej swobody,
  ale płacimy za to utratą ``ładnych'' własności algebraicznych.
\end{frame}

\begin{frame}{Dlaczego w praktyce wygrywają kwaterniony?}
  \begin{itemize}
    \item Mają skończoną, zwartą reprezentację obrotu 3D.
    \item Dają się stabilnie normalizować numerycznie.
    \item Nie mają problemu gimbal lock, w przeciwieństwie do kątów Eulera.
    \item Są wspierane natywnie przez silniki gier (Unity), narzędzia 3D (Blender),
    biblioteki grafiki, kontrolery lotu dronów.
    \item Nadal można je mnożyć i dzielić (czyli mamy sensowną algebrę).
  \end{itemize}

  \vspace{0.3cm}
  \textbf{Wniosek praktyczny:} kwaterniony to nie tylko ciekawostka matematyczna.
  To narzędzie inżynierskie.
\end{frame}

\begin{frame}{Podsumowanie}
  \begin{itemize}
    \item Liczby wielowymiarowe powstają przez kolejne rozszerzenia:
    \[
      \mathbb{R} \rightarrow \mathbb{C} \rightarrow \mathbb{H} \rightarrow \mathbb{O} \rightarrow \dots
    \]
    (konstrukcja Cayley--Dicksona).
    \item Każde rozszerzenie podwaja wymiar, ale \textbf{psuje} jakąś własność
    (najpierw ``porządek'', potem przemienność, potem łączność itd.).
    \item Kwaterniony są kluczowe praktycznie:
      opisują obroty 3D w stabilny numerycznie sposób.
    \item Oktoniony i dalej są głównie ciekawostką teoretyczną
      (fizyka teoretyczna, struktury wyjątkowe), bo tracimy już zbyt wiele własności.
  \end{itemize}
\end{frame}

\begin{frame}{Źródła}
  \scriptsize
  \begin{itemize}
    \item R. Hamilton: prace o kwaternionach (1843).
    \item J. Baez, ``The Octonions'', Bulletin AMS, 2002.
    \item A. Kuipers, ``Quaternions and Rotation Sequences'', Princeton, 1999.
    \item Materiały nt. konstrukcji Cayley--Dicksona (Cayley--Dickson construction).
    \item ``Aksjomaty i konstrukcje liczb'', Wikipedia PL.
    \item Dokumentacja Unity / Blender: reprezentacja rotacji przez kwaterniony.
  \end{itemize}

  \vspace{0.3cm}
  
\end{frame}

% =================================================
\end{document}
